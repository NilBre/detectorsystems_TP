% This example is meant to be compiled with lualatex or xelatex
% The theme itself also supports pdflatex
\PassOptionsToPackage{unicode}{hyperref}
\documentclass[aspectratio=1610, 9pt]{beamer}

% Warning, if another latex run is needed
% \usepackage[aux]{rerunfilecheck}

% just list chapters and sections in the toc, not subsections or smaller
\setcounter{tocdepth}{1}

%------------------------------------------------------------------------------
%------------------------------ Fonts, Unicode, Language ----------------------
%------------------------------------------------------------------------------
\usepackage{fontspec}
\defaultfontfeatures{Ligatures=TeX}  % -- becomes en-dash etc.

% german language
\usepackage{polyglossia}
\setdefaultlanguage{german}

% for english abstract and english titles in the toc
\setotherlanguages{english}

% intelligent quotation marks, language and nesting sensitive
\usepackage[autostyle]{csquotes}

% microtypographical features, makes the text look nicer on the small scale
\usepackage{microtype}

%------------------------------------------------------------------------------
%------------------------ Math Packages and settings --------------------------
%------------------------------------------------------------------------------

\usepackage{amsmath}
\usepackage{amssymb}
\usepackage{mathtools}
\usepackage{bbold}

% Enable Unicode-Math and follow the ISO-Standards for typesetting math
\usepackage[
  math-style=ISO,
  bold-style=ISO,
  sans-style=italic,
  nabla=upright,
  partial=upright,
]{unicode-math}
\setmathfont{Latin Modern Math}

% nice, small fracs for the text with \sfrac{}{}
\usepackage{xfrac}


%------------------------------------------------------------------------------
%---------------------------- Numbers and Units -------------------------------
%------------------------------------------------------------------------------

\usepackage[
  locale=DE,
  separate-uncertainty=true,
  per-mode=symbol-or-fraction,
]{siunitx}
\sisetup{math-micro=\text{µ},text-micro=µ}
% \sisetup{tophrase={{ to }}}
%------------------------------------------------------------------------------
%-------------------------------- tables  -------------------------------------
%------------------------------------------------------------------------------

\usepackage{booktabs}       % \toprule, \midrule, \bottomrule, etc

%------------------------------------------------------------------------------
%-------------------------------- graphics -------------------------------------
%------------------------------------------------------------------------------

\usepackage{graphicx}
%\usepackage{rotating}
\usepackage{grffile}
\usepackage{tikz}
\usepackage{circuitikz}
\usepackage{tikz-feynman}
\usepackage{subcaption}

% allow figures to be placed in the running text by default:
\usepackage{scrhack}
\usepackage{float}
\floatplacement{figure}{htbp}
\floatplacement{table}{htbp}

% keep figures and tables in the section
\usepackage[section, below]{placeins}

% smileys
\usepackage{MnSymbol,wasysym}

%------------------------------------------------------------------------------
%---------------------- customize list environments ---------------------------
%------------------------------------------------------------------------------

\usepackage{enumitem}
\usepackage{listings}
\usepackage{hepunits}

\usepackage{pdfpages}
%------------------------------------------------------------------------------
%------------------------------ Bibliographie ---------------------------------
%------------------------------------------------------------------------------

\usepackage[
  backend=biber,   % use modern biber backend
  autolang=hyphen, % load hyphenation rules for if language of bibentry is not
                   % german, has to be loaded with \setotherlanguages
                   % in the references.bib use langid={en} for english sources
]{biblatex}
\addbibresource{references.bib}  % the bib file to use
\DefineBibliographyStrings{german}{andothers = {{et\,al\adddot}}}  % replace u.a. with et al.


% Load packages you need here
% \usepackage{polyglossia}
% \setmainlanguage{german}

\usepackage{csquotes}


% \usepackage{amsmath}
% \usepackage{amssymb}
% \usepackage{mathtools}

\usepackage{hyperref}
\usepackage{bookmark}

% load the theme after all packages

\usetheme[
  showtotalframes, % show total number of frames in the footline
]{tudo}

% Put settings here, like
\unimathsetup{
  math-style=ISO,
  bold-style=ISO,
  nabla=upright,
  partial=upright,
  mathrm=sym,
}

% \setbeamertemplate{itemize item}{\scriptsize$\blacktriangleright$}
% \setbeamertemplate{itemize subitem}{\scriptsize$\blacktriangleright$}

%Titel:
\title{Gamma-Spectroscopy}
%Autor
\author[N.Breer]{Nils Breer}
%Lehrstuhl/Fakultät
\institute{Fakultät Physik}
%Titelgrafik muss ich einfueren!!!
%\titlegraphic{\includegraphics[width=0.3\textwidth]{content/Bilder/interferenz.jpg}}
\date{12.07.2022}

\begin{document}
\maketitle

\begin{frame}\frametitle{Agenda}
  \begin{itemize}
    \item What is gamma spectroscopy?
    \item Interactions in the Spectrum
    \item Detectorsystems
    \item Applications
    \item Summary
  \end{itemize}
\end{frame}

\begin{frame}\frametitle{What is gamma spectroscopy?}
  \begin{itemize}
    \item \textbf{studies of energy spectra of gamma rays}
    \item identification of gamma-emitting radionuclides
    \item Interactions: Photoeffect, Compton scattering, Pair production
  \end{itemize}
\end{frame}

\begin{frame}\frametitle{WInteraction of $\gamma$-rays with matter}
  \begin{columns}
    \begin{column}[c]{0.48\textwidth}
      \begin{figure}
        \includegraphics[width=\textwidth]{plots/z_depend.png}
        \caption{atomic number Z against photon energy E.}
      \end{figure}
    \end{column}
    \begin{column}[c]{0.48\textwidth}
      \begin{itemize}
        \item processes above ionization threshold
        \item Gamma ray absorption \to intensity loss
        \item Material thickness dependend intensity: $N(D) = N_0 e^{-\mu D}$
        \item D: thickness, $\mu$: absoption coefficient, $N_0$: initial intensity
      \end{itemize}
    \end{column}
  \end{columns}
\end{frame}

\begin{frame}\frametitle{Photoeffect}
  \begin{columns}
    \begin{column}[c]{0.48\textwidth}
      \begin{figure}
        \includegraphics[width=\textwidth]{plots/photo_abs.png}
        \caption{photo peak}
      \end{figure}
    \end{column}
    \begin{column}[c]{0.48\textwidth}
      \begin{itemize}
        \item $E_\gamma < $ several 100 keV
        \item ionizing bound electron (K-shell)
        \item $\gamma + atom \to atom^{+} + e^{-}$
        \item hole is filled with electrons from higher shells recursively
        \item energy diff. release as x-rays characteristic
        \item rarely: photon leaves absorber. often excite more electrons inside
        \item K-L-M-absorption edge: Quantumenergy enough to release bound electron from given shell
      \end{itemize}
    \end{column}
  \end{columns}
\end{frame}

\begin{frame}\frametitle{Compton scattering}
  \begin{columns}
    \begin{column}[c]{0.48\textwidth}
      \begin{figure}
        \includegraphics[width=\textwidth]{plots/compton.png}
        \caption{compton continuum}
      \end{figure}
    \end{column}
    \begin{column}[c]{0.48\textwidth}
      \begin{itemize}
        \item main interaction (100 keV < E < 5 MeV)
        \item inelastic scattering
        \item photons only transfers an energy fraction
        \item cannot view full spectrum \frownie{}
      \end{itemize}
    \end{column}
  \end{columns}
\end{frame}

\begin{frame}\frametitle{Compton scattering}
  \begin{columns}
    \begin{column}[c]{0.48\textwidth}
      \includegraphics[width=\textwidth]{plots/compton_kante.png}
    \end{column}
    \begin{column}[c]{0.48\textwidth}
      \begin{itemize}
        \item non-isotropic angular distribution
        \item $E_\gamma\prime = \frac{E_\gamma}{1 + \left( \frac{E_\gamma}{511 keV} (1 - cos\theta)\right)}$
        \item $E_{e^-} = E_\gamma \left( \frac{\frac{E_\gamma}{511 keV}(1 - cos\theta)}{1 + \frac{E_\gamma}{511 keV}(1 - cos\theta)} \right)$
        \item extreme cases: backward scattering, light graze
      \end{itemize}
    \end{column}
  \end{columns}
\end{frame}

\begin{frame}\frametitle{Pair production}
  \begin{columns}
    \begin{column}[c]{0.48\textwidth}
      \includegraphics[width=\textwidth]{plots/pair_triplett.png}
    \end{column}
    \begin{column}[c]{0.48\textwidth}
      \begin{itemize}
        \item photon produces e+e- pair if E is high enough (5 MeV < E < 10 MeV)
        \item occurs in proximity of nucleus/scattering partner
        % \item in restframe photon mass = 0, lepton mass > 0; someone needs to take the momentum
        \item photon line visible if both leptons are absorbed;
        \item annihilation peak : 511 keV (e- mass) or doubled for both
        \item single- and double-escape peaks
      \end{itemize}
    \end{column}
  \end{columns}
\end{frame}

\begin{frame}\frametitle{Energy spectra}
  \begin{figure}
    \includegraphics[width=0.9\textwidth]{plots/full_spec.png}
    \caption{Full spectrum of Cs-137 source}
  \end{figure}
\end{frame}

\begin{frame}\frametitle{Band structure}
  \begin{columns}
    \begin{column}[c]{0.48\textwidth}
      \includegraphics[width=\textwidth]{plots/bands.png}
    \end{column}
    \begin{column}[c]{0.48\textwidth}
      \begin{itemize}
        \item electrons in discrete/precise energy bands
        \item valence band: outer band for chemical reactions; most inhibited
        \item conduction band: migration of electrons
      \end{itemize}
    \end{column}
  \end{columns}
\end{frame}

\begin{frame}\frametitle{Mobility of "Holes"}
  \begin{columns}
    \begin{column}[c]{0.48\textwidth}
      \includegraphics[width=\textwidth]{plots/holes.png}
    \end{column}
    \begin{column}[c]{0.48\textwidth}
      \begin{itemize}
        \item positive charge $\equiv$ hole in the band
        \item measuring the energy relies on separating the charge carriers
        \item electrons from valence band filling holes -> effective moving
        \item -> conductivity
      \end{itemize}
    \end{column}
  \end{columns}
\end{frame}

\begin{frame}\frametitle{Creation of charge carriers}
  \begin{columns}
    \begin{column}[c]{0.48\textwidth}
      \begin{itemize}
        \item excite electrons from low bands through high energies (E > $E_\text{therm}$)
        \item redistribution of electrons-holes throughout energy-bands
        \item holes: top of valence band
        \item electrons: bottom of conduction band
        \item external field: charge carriers migrate towards respective electrode
      \end{itemize}
    \end{column}
    \begin{column}[c]{0.48\textwidth}
      \begin{itemize}
        \item number of electron-hole pairs $n = E_\text{abs} / \epsilon$
        \item $\epsilon$: average energy needed to create electron hole pair
        \item $E_\text{abs}$: absorbed gamma ray energy
      \end{itemize}
    \end{column}
  \end{columns}
\end{frame}

\begin{frame}\frametitle{resolution and suitable semiconductors}
resolution $\propto$ n -> detector needs specific properties
  \begin{itemize}
    \item large absorption coefficient (high atomic number Z)
    \item low $\epsilon$: to produce many electron-hole pairs
    \item allow good Mobility (trapping inside semiconductor lattice)
    \item pure crystal structure (traps for charge carriers)
    \item cannot be too expensive
  \end{itemize}
\end{frame}

\begin{frame}\frametitle{Why Germanium as detector material?}
  \begin{columns}
    \begin{column}[c]{0.48\textwidth}
      \includegraphics[width=\textwidth]{plots/candidates.png}
    \end{column}
    \begin{column}[c]{0.48\textwidth}
      \begin{itemize}
        \item Silicon: highly pure, low-priced, low atomic number (low energy photons)
        \item Germanium: higher Z -> good for higher energy gamma radiation
        \item improvements reduced resolution to $\approx$ 1.8 keV
        \item low temperature -> reduce leakage current
      \end{itemize}
    \end{column}
  \end{columns}
\end{frame}

\begin{frame}\frametitle{Why Germanium as detector material?}
  \begin{columns}
    \begin{column}[c]{0.48\textwidth}
      \includegraphics[width=\textwidth]{plots/ge.png}
    \end{column}
    \begin{column}[c]{0.48\textwidth}
      \begin{itemize}
        \item
      \end{itemize}
    \end{column}
  \end{columns}
\end{frame}

\begin{frame}\frametitle{Semiconductor detector}
  \begin{columns}
    \begin{column}[c]{0.48\textwidth}
      \includegraphics[width=\textwidth]{plots/junction.png}
    \end{column}
    \begin{column}[c]{0.48\textwidth}
      \begin{itemize}
        \item p-n or M-S junctions possible
        \item electrodes form a metal-semiconductor junction
        \item impurities: p-type (acceptor states), n-type (donator states)
        \item solid detection material for full energy deposition inside
        \item doping: adding energy states -> narrowing the band gap
        % \item electrical properties needed for electron-hole pairs to electric signal
        \item maximising the depletion zone -> hinder recombination -> reconstruct energy of the event
        \item probability of thermal excitation: $p(T) = T^{3/2} exp(-E_g / 2 k_b T)$
      \end{itemize}
    \end{column}
  \end{columns}
\end{frame}

\begin{frame}\frametitle{Metal-Semiconductor junction}
  \begin{columns}
    \begin{column}[c]{0.48\textwidth}
      \includegraphics[width=0.7\textwidth]{plots/pre_m_s.png}
    \end{column}
    \begin{column}[c]{0.48\textwidth}
      \begin{itemize}
        \item $W_0$: vacuum level, $E_L$: conduction band, $E_F$: fermi energy
        \item $E_V$: valence band, $E_{LM}$: energy level in metal, $E_{FM}$: fermi energy in metal
        \item electrons can migrate from Semiconductor to metal since E-level is lower
        \item probability of finding electrons in conduction band gets lower
      \end{itemize}
    \end{column}
  \end{columns}
\end{frame}

\begin{frame}\frametitle{Metal-Semiconductor junction (n-Si)}
  \begin{columns}
    \begin{column}[c]{0.48\textwidth}
      \includegraphics[width=0.7\textwidth]{plots/post_m_s.png}
    \end{column}
    \begin{column}[c]{0.48\textwidth}
      \begin{itemize}
        \item migrated charge carriers form depletion zone and lower fermi level
        \item additional $e^{-}$ must overcome the Schottky-barrier to flow into Metal
        \item fermi levels in metal and semiconductor equalize via diffusion process
      \end{itemize}
    \end{column}
  \end{columns}
\end{frame}

\begin{frame}\frametitle{Why is this junction used over p-n?}
  \begin{itemize}
    \item Si-Schottky diodes are substantially faster changing from forward bias to reverse bias
    \item -> switching action: 10 - 100 GHz possible because no "holes" in metal
    \item low forward voltage drop (0.15 - 0.45V) p-n: 0.6 - 0.75 V
    \item But: higher reverse leakage current
  \end{itemize}
\end{frame}

\begin{frame}\frametitle{How to semiconductor}
  \begin{itemize}
    \item p and n doped areas
    \item charge carriers diffuse and recombinate
    \item surface is charge carrier poor zone
    \item acceptor in p, donator in n. electric field hinders carriers
    \item the bigger the zone the better the seperation -> more $U_g$
    \item seperation before recombination \to pulse \to quatification of energy
    \item only possible if generated in depletion zone (charge carrier poor zone)
    \item depletion zone big -> reverse voltage and doping material
    \item asymmetric doping (equation)
    \item noise: electrons randomly passing reverse voltage, also cool detector
    \item veto region with alu case ($E_{min} = 40 - 50$ keV)
    \item Li in Ge for n, Au in Ge for p
  \end{itemize}
\end{frame}

\begin{frame}\frametitle{Evaluation of gamma spec}
  \begin{columns}
    \begin{column}[c]{0.48\textwidth}
      PRO:
      \begin{itemize}
        \item quite cheap in material costs
        \item relatively fast result evaluation
        \item multinuclide analysis (distinct lines visible for all nuclides)
        \item non-destructive for emitter (radiation hardness of detector given)
        \item remote measurement
      \end{itemize}
    \end{column}
    \begin{column}[c]{0.48\textwidth}
      CONTRA:
      \begin{itemize}
        \item often less sensitive
        \item require large sample masses (if not gamma rays from space)
      \end{itemize}
    \end{column}
  \end{columns}
\end{frame}

\begin{frame}\frametitle{Table of interesting radio nuclides}
  \includegraphics[width=0.8\textwidth]{plots/nuclid_table1.png}
\end{frame}

\begin{frame}\frametitle{Summary}
  \begin{itemize}
    \item
  \end{itemize}
\end{frame}

\begin{frame}\frametitle{Quellen}
\url{https://onlinelibrary.wiley.com/doi/book/10.1002/9780470861981} \\
\url{V18_Anleitung.pdf} \\
\url{https://www.nrc.gov/docs/ML1122/ML11229A699.pdf} \\
\url{https://www.electrical4u.com/schottky-diode/} \\
\end{frame}

\begin{frame}\frametitle{Backup}
  \begin{columns}
    \begin{column}[c]{0.45\textwidth}
      \begin{itemize}
        \item Attenuation egde for caesium iodine (CsI)
        \item 2 K-lines and 2 L-lines
      \end{itemize}
    \end{column}
  \begin{column}[c]{0.45\textwidth}
    \includegraphics[width=0.8\textwidth]{plots/attenuation_edges.png}
  \end{column}
  \end{columns}
\end{frame}

\end{document}
